\documentclass[a4paper]{article}

\usepackage{hyperref}
\usepackage{graphicx}
\usepackage{caption}
\usepackage{subcaption}
\usepackage{kpfonts}
\graphicspath{ {img/} }

\usepackage{amsmath}
\usepackage{bbm}
\usepackage{dsfont}
\usepackage{amsfonts}
\usepackage{amsthm}
\usepackage{mathtools}
\usepackage{gensymb}
\usepackage[margin=1in]{geometry}
\usepackage{tabto}
\usepackage{xcolor}

\geometry{rmargin=1.5in}

\makeatletter
\renewcommand\section{\@startsection{section}{1}{\z@}%
                                  {-3.5ex \@plus -1ex \@minus -.2ex}%
                                  {2.3ex \@plus.2ex}%
                                  {\normalfont\normalsize\bfseries}}
\makeatletter


\title{\Large Parallel Programming Project 1}
\author{VOYLES Evan}
\date{}

\newtheorem{theorem}{Th\'eor\`eme}
\newtheorem*{theorem*}{Th\'eor\`eme}
\newtheorem{exercice}{Exercice}
\newtheorem*{exercice*}{Exercice bonus}
\theoremstyle{definition}
\newtheorem{definition}{D\'efinition}
\newtheorem*{definition*}{D\'efinition}
\newtheorem{lemme}{Lemme}
\newtheorem{proposition}{Proposition}
\newtheorem*{proposition*}{Proposition}

\begin{document}

\makeatletter
\maketitle
\begin{center}
    \vspace{-.2in}
    \today
\end{center}
\makeatother

\newenvironment{norm}{
    \normalfont
}

\section{Preliminaries}

Any sufficiently regular function defined on a sphere can be written
\begin{equation} \label{eq:f_iso}
    f(\theta, \phi) = \sum_{l = 0}^{+\infty}\sum_{m = 0}^l \bar P_l^m(\cos\theta)[C_l^m\cos m\phi + S_l^m \sin m \phi].
\end{equation}

Where $C_l^m, S_l^m$ are \textit{spherical harmonic} coefficients (also referred to as Laplace series coefficients) of 
degree \textit{l} and order \textit{m}; $\bar P_l^m$ are the fully normalized Associated Legendre Functions of degree $l$ and order $m$;
$\theta$ is the polar angle and $\phi$ is the azimuthal angle, as per ISO convention\footnote{We have chosen the ISO standard coordinate system to 
ensure the integrity of future computations. For a discussion on transforming the $(\phi, \lambda)$ coordinates provided by NASA, see Appendix A}.

The project proposal suggests computing the coefficients $C_l^m, S_l^m$ by setting up an overdetermined linear system of equations and finding the least squares
solution. Unfortunately, this problem runs in $O(Nl_{max}^4)$ time and requires about $8N(l_{max} + 1)^2$ bytes of memory. That is absurdly prohibitive. Furthermore, there 
are no \textit{self-evident} parallelization techniques to speed up computation time. In this report we propose an alternative approach to computing the Laplace series coefficients
$C_l^m$ and $S_l^m$ that runs in \textcolor{red}{$O(Nl_{max}^2)$} time complexity with trivial parallelization models.

\section{Alternative Approach}

The coefficients $C_l^m$ and $S_l^m$ are defined as the orthogonal projection of $f_(\theta, \phi)$ onto the basis functions (spherical harmonics $Y_l^m$). Written in terms of the real
spherical harmonics:

\begin{align*}
    C_l^m &= \int_0^{2\pi}\int_0^\pi f(\theta, \phi) P_l^m (\cos \theta)\cos (m \phi) \sin \theta d\theta d\phi\\
    S_l^m &= \int_0^{2\pi}\int_0^\pi f(\theta, \phi) P_l^m (\cos \theta)\sin (m \phi) \sin \theta d\theta d\phi
\end{align*}

\subsection{Numeric Integraion}

We first remark that the NASA data sets are provided as discretizations of $[-\frac{\pi}{2}, \frac{\pi}{2}] \times [-\pi, \pi]$, represented as a $n_\theta \times n_\phi$ equidistant grid.
For example, for \verb|ETOPO1_small.csv|, $n_\theta = 180$, $n_\phi = 360$, and $N \coloneqq n_\theta n_\phi = 64000$;
We can easily approximate these 2 dimensional integrals (using the midpoint rule?) via numeric quadrature rules
\begin{align*}
    C_l^m &\approx \sum_{i = 1}^{n_\phi}\sum_{j = 1}^{n_\theta} f(\theta_j, \phi_i) P_l^m (\cos \theta_j)\cos (m \phi_i) \sin \theta_j \Delta\theta \Delta\phi\\
          &\approx \Delta\theta \Delta\phi\sum_{i = 1}^{n_\phi}\cos (m \phi_i)\sum_{j = 1}^{n_\theta} f(\theta_j, \phi_i) P_l^m (\cos \theta_j) \sin \theta_j\\
\end{align*}

\begin{align*}
    S_l^m &\approx \sum_{i = 1}^{n_\phi}\sum_{j = 1}^{n_\theta} f(\theta_j, \phi_i) P_l^m (\cos \theta_j)\sin (m \phi_i) \sin \theta_j \Delta\theta \Delta\phi\\
          &\approx \Delta\theta \Delta\phi\sum_{i = 1}^{n_\phi}\sin (m \phi_i)\sum_{j = 1}^{n_\theta} f(\theta_j, \phi_i) P_l^m (\cos \theta_j) \sin \theta_j\\
\end{align*}

This method computes \textit{individual} Laplace series coefficients in $O(N)$ time! Thus, computing the coefficients for a Discrete Laplace Series (DLS) of order $l_max$ runs in
$O(Nl^2_{max})$. More interestingly, each $C_l^m$ or $S_l^m$ can be computed \textbf{independently} of any other coefficient. Whereas solving the linear least squares problem requires computing
every $C_l^m$ and $C_l^m$ in a single operation, this method allows us the ability to compute any arbitrary $C_l^m$ directly. Why do we care? If I alraedy have the first 500 $C_l^m$ coefficients computed
and stored in a text file, I can \textit{directly} compute the 501st in $O(N)$ time. Furthermore, this method gives rise to obvious and easy-to-implement parallelization schemes.

\subsection{Comparison with Linear Squares}

Tested on the \verb|ETOPO1_small.csv| data set with $l_{max} = 20$, the model computed via numerical integration had an average error of about 2 times that of the least squares approach, but ran over 100 times faster.

\textbf{TODO}
\begin{itemize}
    \item Provide more detailed numeric comparison
    \item Provide timing analysis
\end{itemize}

\section{Improvements to the model}

The least squares method provides a more accurate model, but runs in $O(Nl^4_{max})$ time. In this section we propose improvements to our model computed via quadrature.

\subsection{Gradient Descent}

We can apply methods of optimization to minizmize the mean squared error (MSE) of our model. To begin we explicitze the MSE then compute its gradient



\textbf{TODO} I have already derived the gradient of the MSE with respect to all $C_l^m$ and $S_l^m$ on paper
\begin{itemize}
    \item Write up derivation in latex
\end{itemize}

Computing the gradient vector runs in \textcolor{red}{$O(Nl_{max}^2)$} time


\subsubsection{Stochastic Gradient Descent}

Computing the gradient with respect to a single $C_l^m$ requires summing up certain values for all $N$ points of the data set. For larger data sets (e.g. $N_{high} = 1800 * 3600 = 6480000$), this
computation can become intractable. We can speed up our gradient computation by computing as estimate of the gradient, $\hat\nabla$MSE, by randomly sampling $n$ points from our entire
data set. This reduces the computation of a gradient vector from $O(Nl_{max}^2)$ to \textcolor{red}{$O(nl_{max}^2)$}. This is an \textbf{insane} improvement to the run time of our algorithm.
Instead of summing all 6480000 points of the large data set, we could fix $n$ at say $n = 1000$ instead, sacrificing accuracy for a speedeup of 6,480.

\subsection{MCMC Algorithms}

In initial tests my gradient descent learning is \textit{sometimes} reducing the MSE but other times it's increasing it... If I don't figure out exactly what's going wrong 
then I plan on implementing MCMC algorithms like Gibb's sampling to explore the sample space and only accept proposal vectors that decrease the MSE.

\section{Parallelization} 

\textbf{TODO} (trivial)

\section{Conclusion}

Instead of calculating the model with a direct method that runs in $O(Nl_{max}^4)$ time, We propose a hybrid model that first directly computes the coefficients in $O(Nl_{max}^2)$ time then 
iteratively improves the MSE via stochastic methods running in $O(n_{training}n_{sample}l_{max}^2)$ time.

Results are to follow :)





\appendix
\section{Spherical Coordinates Transformation}

The data sets \verb|ETOPO1_*.csv| provided by NASA represent spherical coordinates in an unconventional format: $$(\phi, \lambda) \in [-\frac{\pi}{2}, \frac{\pi}{2}] \times [-\pi, \pi] $$ where $\phi$ refers to the latitude
and $\lambda$ refers to the longitude. In this section we outline the transformation from NASA's coordinate system to the ISO standard representation and prove equivalence between (1) and the Project presentation's $f(\phi, \lambda)$.

To avoid confusion, we denote the spherical coordinate pair in ISO coordinates as $(\theta_{iso}, \phi_{iso})$ and NASA's coordinate scheme as $(\phi_{nasa}, \lambda_{nasa})$.
We remark that the polar angle $\theta_{iso}$ is simply equal to $\frac{\pi}{2} - \phi_{nasa}$. We corroborate this fact by remarking that a latitude of $\phi_{nasa} = 0$ corresponds to a polar
angle of $\phi_{iso} = \frac{\pi}{2}$. Conveniently, the longitude $\lambda_{nasa}$ is equivalent to the azimuthal angle $\phi_{iso}$.

\begin{proposition} Let the spherial coordinate pairs $(\theta_{iso}, \phi_{iso})$, $(\phi_{nasa}, \lambda_{nasa})$ be defined as above. 
    \begin{equation*}
        f(\theta_{iso}, \phi_{iso}) \equiv f(\phi_{nasa}, \lambda_{nasa})
    \end{equation*}
    
\end{proposition}

\begin{proof}
    $f(\phi_{nasa}, \lambda_{nasa})$ is defined in the Project proposal as 
    \begin{align} \label{eq:project_def} 
        f(\phi_{nasa}, \lambda_{nasa}) &= \sum_{l = 0}^{+\infty}\sum_{m = 0}^l \bar P_l^m(\sin\phi_{nasa})[C_l^m\cos m\lambda_{nasa} + S_l^m \sin m \lambda_{nasa}] \\ 
                                       &= \sum_{l = 0}^{+\infty}\sum_{m = 0}^l \bar P_l^m(\sin(\pi/2 - \theta_{iso}))[C_l^m\cos m\phi_{iso} + S_l^m \sin m \phi_{iso}] \\
                                       &= \sum_{l = 0}^{+\infty}\sum_{m = 0}^l \bar P_l^m(\cos\theta_{iso}))[C_l^m\cos m\phi_{iso} + S_l^m \sin m \phi_{iso}] \\
                                       &\equiv f(\theta_{iso}, \phi_{iso})
    \end{align}

\end{proof}



\end{document}

